% Generated by Sphinx.
\def\sphinxdocclass{report}
\newif\ifsphinxKeepOldNames \sphinxKeepOldNamestrue
\documentclass[letterpaper,12pt,english]{sphinxmanual}
\usepackage{iftex}

\ifPDFTeX
  \usepackage[utf8]{inputenc}
\fi
\ifdefined\DeclareUnicodeCharacter
  \DeclareUnicodeCharacter{00A0}{\nobreakspace}
\fi
\usepackage{cmap}
\usepackage[T1]{fontenc}
\usepackage{amsmath,amssymb,amstext}
\usepackage{babel}
\usepackage{times}
\usepackage[Sonny]{fncychap}
\usepackage{longtable}
\usepackage{sphinx}
\usepackage{multirow}
\usepackage{eqparbox}


\addto\captionsenglish{\renewcommand{\figurename}{图 }}
\addto\captionsenglish{\renewcommand{\tablename}{表 }}
\SetupFloatingEnvironment{literal-block}{name=列表 }

\addto\extrasenglish{\def\pageautorefname{page}}

\setcounter{tocdepth}{1}

    \usepackage{xeCJK}
    \usepackage{indentfirst}
    \setlength{\parindent}{2em}
    \setCJKmainfont[BoldFont=SimHei, ItalicFont=FangSong]{SimSun}
    \setCJKmonofont[Scale=0.9]{FangSong}
    \setCJKfamilyfont{song}[BoldFont=SimSun]{SimSun}
    \setCJKfamilyfont{sf}[BoldFont=SimSun]{SimSun}
    \XeTeXlinebreaklocale "zh"
    \XeTeXlinebreakskip = 0pt plus 1pt
    

\title{tio2 book Documentation}
\date{9月 15, 2016}
\release{0.0.1}
\author{tio2}
\newcommand{\sphinxlogo}{}
\renewcommand{\releasename}{发布}
\makeindex

\makeatletter
\def\PYG@reset{\let\PYG@it=\relax \let\PYG@bf=\relax%
    \let\PYG@ul=\relax \let\PYG@tc=\relax%
    \let\PYG@bc=\relax \let\PYG@ff=\relax}
\def\PYG@tok#1{\csname PYG@tok@#1\endcsname}
\def\PYG@toks#1+{\ifx\relax#1\empty\else%
    \PYG@tok{#1}\expandafter\PYG@toks\fi}
\def\PYG@do#1{\PYG@bc{\PYG@tc{\PYG@ul{%
    \PYG@it{\PYG@bf{\PYG@ff{#1}}}}}}}
\def\PYG#1#2{\PYG@reset\PYG@toks#1+\relax+\PYG@do{#2}}

\expandafter\def\csname PYG@tok@gd\endcsname{\def\PYG@tc##1{\textcolor[rgb]{0.63,0.00,0.00}{##1}}}
\expandafter\def\csname PYG@tok@gu\endcsname{\let\PYG@bf=\textbf\def\PYG@tc##1{\textcolor[rgb]{0.50,0.00,0.50}{##1}}}
\expandafter\def\csname PYG@tok@gt\endcsname{\def\PYG@tc##1{\textcolor[rgb]{0.00,0.27,0.87}{##1}}}
\expandafter\def\csname PYG@tok@gs\endcsname{\let\PYG@bf=\textbf}
\expandafter\def\csname PYG@tok@gr\endcsname{\def\PYG@tc##1{\textcolor[rgb]{1.00,0.00,0.00}{##1}}}
\expandafter\def\csname PYG@tok@cm\endcsname{\let\PYG@it=\textit\def\PYG@tc##1{\textcolor[rgb]{0.25,0.50,0.56}{##1}}}
\expandafter\def\csname PYG@tok@vg\endcsname{\def\PYG@tc##1{\textcolor[rgb]{0.73,0.38,0.84}{##1}}}
\expandafter\def\csname PYG@tok@vi\endcsname{\def\PYG@tc##1{\textcolor[rgb]{0.73,0.38,0.84}{##1}}}
\expandafter\def\csname PYG@tok@mh\endcsname{\def\PYG@tc##1{\textcolor[rgb]{0.13,0.50,0.31}{##1}}}
\expandafter\def\csname PYG@tok@cs\endcsname{\def\PYG@tc##1{\textcolor[rgb]{0.25,0.50,0.56}{##1}}\def\PYG@bc##1{\setlength{\fboxsep}{0pt}\colorbox[rgb]{1.00,0.94,0.94}{\strut ##1}}}
\expandafter\def\csname PYG@tok@ge\endcsname{\let\PYG@it=\textit}
\expandafter\def\csname PYG@tok@vc\endcsname{\def\PYG@tc##1{\textcolor[rgb]{0.73,0.38,0.84}{##1}}}
\expandafter\def\csname PYG@tok@il\endcsname{\def\PYG@tc##1{\textcolor[rgb]{0.13,0.50,0.31}{##1}}}
\expandafter\def\csname PYG@tok@go\endcsname{\def\PYG@tc##1{\textcolor[rgb]{0.20,0.20,0.20}{##1}}}
\expandafter\def\csname PYG@tok@cp\endcsname{\def\PYG@tc##1{\textcolor[rgb]{0.00,0.44,0.13}{##1}}}
\expandafter\def\csname PYG@tok@gi\endcsname{\def\PYG@tc##1{\textcolor[rgb]{0.00,0.63,0.00}{##1}}}
\expandafter\def\csname PYG@tok@gh\endcsname{\let\PYG@bf=\textbf\def\PYG@tc##1{\textcolor[rgb]{0.00,0.00,0.50}{##1}}}
\expandafter\def\csname PYG@tok@ni\endcsname{\let\PYG@bf=\textbf\def\PYG@tc##1{\textcolor[rgb]{0.84,0.33,0.22}{##1}}}
\expandafter\def\csname PYG@tok@nl\endcsname{\let\PYG@bf=\textbf\def\PYG@tc##1{\textcolor[rgb]{0.00,0.13,0.44}{##1}}}
\expandafter\def\csname PYG@tok@nn\endcsname{\let\PYG@bf=\textbf\def\PYG@tc##1{\textcolor[rgb]{0.05,0.52,0.71}{##1}}}
\expandafter\def\csname PYG@tok@no\endcsname{\def\PYG@tc##1{\textcolor[rgb]{0.38,0.68,0.84}{##1}}}
\expandafter\def\csname PYG@tok@na\endcsname{\def\PYG@tc##1{\textcolor[rgb]{0.25,0.44,0.63}{##1}}}
\expandafter\def\csname PYG@tok@nb\endcsname{\def\PYG@tc##1{\textcolor[rgb]{0.00,0.44,0.13}{##1}}}
\expandafter\def\csname PYG@tok@nc\endcsname{\let\PYG@bf=\textbf\def\PYG@tc##1{\textcolor[rgb]{0.05,0.52,0.71}{##1}}}
\expandafter\def\csname PYG@tok@nd\endcsname{\let\PYG@bf=\textbf\def\PYG@tc##1{\textcolor[rgb]{0.33,0.33,0.33}{##1}}}
\expandafter\def\csname PYG@tok@ne\endcsname{\def\PYG@tc##1{\textcolor[rgb]{0.00,0.44,0.13}{##1}}}
\expandafter\def\csname PYG@tok@nf\endcsname{\def\PYG@tc##1{\textcolor[rgb]{0.02,0.16,0.49}{##1}}}
\expandafter\def\csname PYG@tok@si\endcsname{\let\PYG@it=\textit\def\PYG@tc##1{\textcolor[rgb]{0.44,0.63,0.82}{##1}}}
\expandafter\def\csname PYG@tok@s2\endcsname{\def\PYG@tc##1{\textcolor[rgb]{0.25,0.44,0.63}{##1}}}
\expandafter\def\csname PYG@tok@nt\endcsname{\let\PYG@bf=\textbf\def\PYG@tc##1{\textcolor[rgb]{0.02,0.16,0.45}{##1}}}
\expandafter\def\csname PYG@tok@nv\endcsname{\def\PYG@tc##1{\textcolor[rgb]{0.73,0.38,0.84}{##1}}}
\expandafter\def\csname PYG@tok@s1\endcsname{\def\PYG@tc##1{\textcolor[rgb]{0.25,0.44,0.63}{##1}}}
\expandafter\def\csname PYG@tok@ch\endcsname{\let\PYG@it=\textit\def\PYG@tc##1{\textcolor[rgb]{0.25,0.50,0.56}{##1}}}
\expandafter\def\csname PYG@tok@m\endcsname{\def\PYG@tc##1{\textcolor[rgb]{0.13,0.50,0.31}{##1}}}
\expandafter\def\csname PYG@tok@gp\endcsname{\let\PYG@bf=\textbf\def\PYG@tc##1{\textcolor[rgb]{0.78,0.36,0.04}{##1}}}
\expandafter\def\csname PYG@tok@sh\endcsname{\def\PYG@tc##1{\textcolor[rgb]{0.25,0.44,0.63}{##1}}}
\expandafter\def\csname PYG@tok@ow\endcsname{\let\PYG@bf=\textbf\def\PYG@tc##1{\textcolor[rgb]{0.00,0.44,0.13}{##1}}}
\expandafter\def\csname PYG@tok@sx\endcsname{\def\PYG@tc##1{\textcolor[rgb]{0.78,0.36,0.04}{##1}}}
\expandafter\def\csname PYG@tok@bp\endcsname{\def\PYG@tc##1{\textcolor[rgb]{0.00,0.44,0.13}{##1}}}
\expandafter\def\csname PYG@tok@c1\endcsname{\let\PYG@it=\textit\def\PYG@tc##1{\textcolor[rgb]{0.25,0.50,0.56}{##1}}}
\expandafter\def\csname PYG@tok@o\endcsname{\def\PYG@tc##1{\textcolor[rgb]{0.40,0.40,0.40}{##1}}}
\expandafter\def\csname PYG@tok@kc\endcsname{\let\PYG@bf=\textbf\def\PYG@tc##1{\textcolor[rgb]{0.00,0.44,0.13}{##1}}}
\expandafter\def\csname PYG@tok@c\endcsname{\let\PYG@it=\textit\def\PYG@tc##1{\textcolor[rgb]{0.25,0.50,0.56}{##1}}}
\expandafter\def\csname PYG@tok@mf\endcsname{\def\PYG@tc##1{\textcolor[rgb]{0.13,0.50,0.31}{##1}}}
\expandafter\def\csname PYG@tok@err\endcsname{\def\PYG@bc##1{\setlength{\fboxsep}{0pt}\fcolorbox[rgb]{1.00,0.00,0.00}{1,1,1}{\strut ##1}}}
\expandafter\def\csname PYG@tok@mb\endcsname{\def\PYG@tc##1{\textcolor[rgb]{0.13,0.50,0.31}{##1}}}
\expandafter\def\csname PYG@tok@ss\endcsname{\def\PYG@tc##1{\textcolor[rgb]{0.32,0.47,0.09}{##1}}}
\expandafter\def\csname PYG@tok@sr\endcsname{\def\PYG@tc##1{\textcolor[rgb]{0.14,0.33,0.53}{##1}}}
\expandafter\def\csname PYG@tok@mo\endcsname{\def\PYG@tc##1{\textcolor[rgb]{0.13,0.50,0.31}{##1}}}
\expandafter\def\csname PYG@tok@kd\endcsname{\let\PYG@bf=\textbf\def\PYG@tc##1{\textcolor[rgb]{0.00,0.44,0.13}{##1}}}
\expandafter\def\csname PYG@tok@mi\endcsname{\def\PYG@tc##1{\textcolor[rgb]{0.13,0.50,0.31}{##1}}}
\expandafter\def\csname PYG@tok@kn\endcsname{\let\PYG@bf=\textbf\def\PYG@tc##1{\textcolor[rgb]{0.00,0.44,0.13}{##1}}}
\expandafter\def\csname PYG@tok@cpf\endcsname{\let\PYG@it=\textit\def\PYG@tc##1{\textcolor[rgb]{0.25,0.50,0.56}{##1}}}
\expandafter\def\csname PYG@tok@kr\endcsname{\let\PYG@bf=\textbf\def\PYG@tc##1{\textcolor[rgb]{0.00,0.44,0.13}{##1}}}
\expandafter\def\csname PYG@tok@s\endcsname{\def\PYG@tc##1{\textcolor[rgb]{0.25,0.44,0.63}{##1}}}
\expandafter\def\csname PYG@tok@kp\endcsname{\def\PYG@tc##1{\textcolor[rgb]{0.00,0.44,0.13}{##1}}}
\expandafter\def\csname PYG@tok@w\endcsname{\def\PYG@tc##1{\textcolor[rgb]{0.73,0.73,0.73}{##1}}}
\expandafter\def\csname PYG@tok@kt\endcsname{\def\PYG@tc##1{\textcolor[rgb]{0.56,0.13,0.00}{##1}}}
\expandafter\def\csname PYG@tok@sc\endcsname{\def\PYG@tc##1{\textcolor[rgb]{0.25,0.44,0.63}{##1}}}
\expandafter\def\csname PYG@tok@sb\endcsname{\def\PYG@tc##1{\textcolor[rgb]{0.25,0.44,0.63}{##1}}}
\expandafter\def\csname PYG@tok@k\endcsname{\let\PYG@bf=\textbf\def\PYG@tc##1{\textcolor[rgb]{0.00,0.44,0.13}{##1}}}
\expandafter\def\csname PYG@tok@se\endcsname{\let\PYG@bf=\textbf\def\PYG@tc##1{\textcolor[rgb]{0.25,0.44,0.63}{##1}}}
\expandafter\def\csname PYG@tok@sd\endcsname{\let\PYG@it=\textit\def\PYG@tc##1{\textcolor[rgb]{0.25,0.44,0.63}{##1}}}

\def\PYGZbs{\char`\\}
\def\PYGZus{\char`\_}
\def\PYGZob{\char`\{}
\def\PYGZcb{\char`\}}
\def\PYGZca{\char`\^}
\def\PYGZam{\char`\&}
\def\PYGZlt{\char`\<}
\def\PYGZgt{\char`\>}
\def\PYGZsh{\char`\#}
\def\PYGZpc{\char`\%}
\def\PYGZdl{\char`\$}
\def\PYGZhy{\char`\-}
\def\PYGZsq{\char`\'}
\def\PYGZdq{\char`\"}
\def\PYGZti{\char`\~}
% for compatibility with earlier versions
\def\PYGZat{@}
\def\PYGZlb{[}
\def\PYGZrb{]}
\makeatother

\renewcommand\PYGZsq{\textquotesingle}

\begin{document}

\maketitle
\tableofcontents
\phantomsection\label{index::doc}


This book is meant to be a collention of investing stories and stories for different companies such as biotech pharmaceutical companies.

Contents:


\chapter{Robert Duggan 中文}
\label{chapters/chapter1:welcome-to-tio2-book-s-documentation}\label{chapters/chapter1::doc}\label{chapters/chapter1:robert-duggan}\label{chapters/chapter1:robertduggan}
测试中文

Robert Duggan, CEO of biotech drugmaker Pharmacyclics PCYC,  will pocket over \$3.5 billion from the company’s sale to AbbVie ABBV, one of the biggest paydays ever from the buyout of a publicly held company.

Under terms of the deal, Duggan will receive \$3.55 billion for his 13.6 million shares, about an 18\% stake in the biotech company, according to corporate filings.

Other senior execs are set for big payouts as well, including Chief Operating Officer Mahkam Zanganeh, whose shares are worth nearly \$224.6 million, and director David Smith, who will pocket nearly \$46.5 million. All told, directors and senior executives could  receive nearly \$4 billion in merger-related payments, Pharmacyclics says.

Duggan, 70, is a longtime private venture investor and was CEO of surgical systems maker Computer Motion until it was acquired by Intuitive Surgical  ISRG in 2003.

Duggan began buying Pharmacyclics stock in 2004, eventually amassing nearly a 25\% stake. But  by 2008, shares had fallen below \$1.  The company’s fortunes turned on chronic lymphocytic leukemia treatment Imbruvica and a series of drug industry mergers.

Pharmacyclics shares currently trade at about \$258.  Takeover speculation – including interest  from Imbruvica partner Johnson \& Johnson  JNJ – have boosted Pharmacyclics 119\%  this year alone, based on Tuesday’s \$257.75 close.

Duggan has declined compensation from the company since becoming CEO in 2008.


\chapter{Accel}
\label{chapters/chapter2::doc}\label{chapters/chapter2:accel}\label{chapters/chapter2:id1}\begin{quote}

A few months after struggling to raise a new fund in 2005, Accel Partners bet \$12.2 million on a website run by a college dropout. Seven years later, that wager is poised to be the most profitable ever for a venture firm.
\end{quote}

Accel, whose partners include Jim Breyer and Kevin Efrusy, is the top outside investor in Facebook Inc., owning about 10 percent. Assuming Facebook is valued at \$100 billion, Accel’s stake on paper is worth about \$10 billion.

When Accel made its Facebook investment, the site had just 2.8 million users -- all on college campuses -- and was run by a 21-year-old Mark Zuckerberg. Now it has 800 million members worldwide and an estimated \$4.27 billion in 2011 sales, according to EMarketer Inc. That explosive growth is poised to deliver an 800-fold return on Accel’s money, catapulting the firm to the forefront of the venture industry.

“This is what makes venture capital and Silicon Valley unique in the world,” said Steve Blank, who helped found eight companies and now teaches entrepreneurship at the University of California at Berkeley and Stanford University. “It’s what VCs do incredibly well. Accel, in this one, deserves all of it.”

Accel’s prospective payday shows the hits-driven nature of the venture business, where one investment can make an entire fund profitable and establish a firm as the market leader. Kleiner Perkins Caufield \& Byers had that distinction from early bets on e-commerce and Web search companies, before missing out on most of the social-media leaders. It later bought shares of Facebook and Twitter Inc. at less favorable prices.
IPO Plans

Facebook plans to raise \$10 billion in the IPO, with a filing coming soon, a person with knowledge of the matter said in November. The offering would value the Menlo Park, California-based company at more than \$100 billion, according to the person.

Accel, located a 10-minute drive away from Facebook in Palo Alto, isn’t the lone winner among investors. Russia’s Digital Sky Technologies and PayPal Inc. co-founder Peter Thiel are due for a payback too. Greylock Partners and Meritech Capital Partners also are investors, as well as Elevation Partners, co-founded by U2’s Bono.

Founded in 1983 by Arthur Patterson and Jim Swartz, Accel was an unlikely investor in Facebook. The firm focused on hardware companies such as UUNet Technologies Inc. and Redback Networks Inc. rather than Internet companies during the dot-com boom of the late 1990s. Google Inc., Amazon.com Inc., Yahoo! Inc. and Netscape Communications Corp. went on to produce billions of dollars for other venture backers.
Early Investment

Accel began raising money for a more Web-focused fund in the mid-2000s, though that effort faced challenges. Harvard University and other prospective investors backed out of the fund, forcing Accel to cut the size of it to \$440 million -- smaller than any investment pool it had raised since 1998.

In May 2005, less than six months after completing the fundraising, Accel made one of the first investments: a startup that was then called Thefacebook. Zuckerberg had begun the company during the previous year in his Harvard dorm room.

Breyer and Efrusy oversaw Accel’s \$12.2 million investment, which assumed a \$100 million valuation, and Breyer gained a seat on the Facebook board.

“It was a big leap of faith at the time,” Efrusy said in an interview in March. “Any one of these investments you do, in hindsight they may look obvious, but at the time they look scary.”

Accel declined to comment for this story, as did Jonathan Thaw, a spokesman for Facebook.
Fund’s Payback

Investors in Accel’s fund have already seen some of the rewards. By selling 17 percent of its original Facebook stake last year at a \$34 billion valuation, Accel paid back investors in the 2005 fund, while holding onto the bulk of its stake for future gains.

Accel’s potential payday would generate more than twice the combined gains of Sequoia Capital and Kleiner Perkins in Google’s 2004 IPO, the biggest venture-backed offering until now, according to the National Venture Capital Association.

Still, Accel’s fortunes will depend on what happens to Facebook’s stock after its IPO. While the venture firm may sell some shares in the offering, it will probably be required to hold the rest for six months, the so-called lock-up period.

Google shares more than doubled in the first six months after the Mountain View, California-based company’s IPO, boosting the value of Sequoia’s and Kleiner Perkins’s holdings. In the 1990s, Web companies such as Yahoo and Amazon also gained most of their value after their IPOs.
New Model

At \$100 billion, Facebook would already be the eighth-biggest U.S. technology company by market value. It would be half the size of Google, even with only 15 percent of the revenue. Facebook stayed private longer than its Internet predecessors, meaning the bulk of investors’ gains may have already been realized, said venture capitalist Maha Ibrahim.

“Facebook has changed the model a little bit,” said Ibrahim, a partner at Canaan Partners in Menlo Park. “Before, it was very rare for a company of their size to still be private.”

As the social-networking market in the U.S. matures, Accel’s partners are scouring the globe for new deals. Breyer is handling investments in China, while Efrusy and Andrew Braccia mine startups in Brazil. Other partners are focusing on Europe, India and Australia.
Groupon Deal

Accel’s recent success isn’t limited to Facebook. The firm was among the biggest venture winners last year, thanks to its stake in daily-deal site Groupon Inc. In last year’s U.S. IPO market, Accel ranked behind New Enterprise Associates, another Groupon investor, and Sequoia, the biggest venture backer of LinkedIn Corp.

Most of Accel’s companies and venture investments in general produce less dramatic results. Coremetrics, a software company backed by Accel, was acquired in 2010 for an undisclosed price, and semiconductor maker Artimi Inc. merged with a competitor in 2008.

Even if Facebook generates a record return for the venture industry, it’s unlikely to spur other startups to file for IPOs soon, said Mark Heesen, president of the National Venture Capital Association in Arlington, Virginia. Facebook’s size and brand are unmatched by any other private, venture-backed company, he said this month in an interview on “Bloomberg West.”

“I don’t think Facebook will have these huge coattails that people are saying it will have,” Heesen said. “Every other possible IPO that’s out there will have to stand on its own.”


\chapter{Tesaro Inc}
\label{chapters/chapter3:tesaro-inc}\label{chapters/chapter3:tesaro}\label{chapters/chapter3::doc}
\emph{Tesaro} was founded in 2010 by a group of pharma industry heavyweights. Rather than spending the time and money to discover drugs, Tesaro acquires compounds from other companies and directs its resources toward the clinical trials needed for FDA approval. Tesaro acquired world-wide rights to niraparib through a 2012 licensing deal with Merck (NYSE: MRK).

TESARO is an oncology-focused biopharmaceutical company devoted to providing transformative therapies to people bravely facing cancer. For more information, visit \url{http://www.tesarobio.com/}, and follow us on Twitter and LinkedIn.

TESARO's Niraparib Significantly Improved Progression-Free Survival for Patients With Ovarian Cancer in Both Cohorts of the Phase 3 NOVA Trial
\begin{itemize}
\item {} 
The NOVA trial successfully achieved its primary endpoint of PFS in the germline BRCA mutant cohort

\item {} 
The NOVA trial successfully achieved its primary endpoint of PFS in the non-germline BRCA mutant cohort, including both the HRD-positive and overall analysis populations

\item {} 
NOVA is the first successful prospectively designed Phase 3 trial of a PARP inhibitor

\item {} 
NDA and MAA submissions are planned for Q4 2016

\end{itemize}
\begin{figure}[htbp]
\centering

\noindent\sphinxincludegraphics{{tesaro1}.jpeg}
\end{figure}
\begin{figure}[htbp]
\centering

\noindent\sphinxincludegraphics{{tesaro}.jpeg}
\end{figure}

WALTHAM, Mass., June 29, 2016 (GLOBE NEWSWIRE) -- TESARO, Inc. (NASDAQ:TSRO), an oncology-focused biopharmaceutical company, today announced that the Phase 3 NOVA trial of niraparib successfully achieved its primary endpoint of progression-free survival (PFS). This trial demonstrated that niraparib significantly prolonged PFS compared to control among patients who are germline BRCA mutation (gBRCAmut) carriers, among patients who are not germline BRCA mutation (non-gBRCAmut) carriers but who have homologous recombination deficient (HRD) tumors as determined by the Myriad myChoice® HRD test, and overall in patients who are not germline BRCA mutation carriers.

An infographic accompanying this release is available at:
\url{http://www.globenewswire.com/NewsRoom/AttachmentNg/f065a7ab-070d-4dc3-8c87-ae4fa8deabd1}

Videos accompanying this release are available at:
\url{http://www.globenewswire.com/NewsRoom/AttachmentNg/39793bb6-2551-47fa-85a3-b19c97d41e11}

\url{http://www.globenewswire.com/NewsRoom/AttachmentNg/6ea284b2-a663-4aeb-96c1-22ac847b460f}

NOVA is a double-blind, placebo-controlled, international Phase 3 trial of niraparib that enrolled more than 500 patients with recurrent ovarian cancer who were in a response to their most recent platinum-based chemotherapy. There is currently no therapy approved by the U.S. Food and Drug Administration for maintenance treatment of patients with recurrent ovarian cancer following response to platinum.

``We are extremely grateful to the patients, caregivers, and investigators who participated in the NOVA trial. The results of this study, which is the first successful, prospectively designed, randomized, well-controlled Phase 3 study of a PARP inhibitor, demonstrate that a single, daily, oral dose of niraparib is superior to control in prolonging PFS in women with recurrent ovarian cancer,'' said Mary Lynne Hedley, Ph.D., President and COO of TESARO. ``Importantly, these results show activity of niraparib in a population of ovarian cancer patients beyond those with germline BRCA mutations. In keeping with our mission of responsible drug development, NOVA was designed to define those patients most likely to benefit from niraparib treatment and, in so doing, optimize the benefit/risk profile for patients. We believe we have achieved that goal and look forward to presentation of the full data set from this study at the European Society for Medical Oncology (ESMO) congress in October.''

Statistically Significant PFS Results in the gBRCAmut Cohort
Among patients who were germline BRCA mutation carriers, the niraparib arm successfully achieved statistical significance over the control arm for the primary endpoint of PFS, with a hazard ratio of 0.27. The median PFS for patients treated with niraparib was 21.0 months, compared to 5.5 months for control (p \textless{} 0.0001).

Statistically Significant PFS Results in non-gBRCAmut Cohort for Patients with HRD Positive Tumors
For patients who were not germline BRCA mutation carriers but whose tumors were determined to be HRD positive using the Myriad myChoice® HRD test, the niraparib arm successfully achieved statistical significance over the control arm for the primary endpoint of PFS, with a hazard ratio of 0.38. The median PFS for patients with HRD-positive tumors who were treated with niraparib was 12.9 months, compared to 3.8 months for control (p \textless{} 0.0001).

Statistically Significant PFS Results in the Overall non-gBRCAmut Cohort
Niraparib also showed statistical significance in the overall non-germline BRCA mutant cohort, which included patients with both HRD-positive and HRD-negative tumors. The niraparib arm successfully achieved statistical significance over the control arm for the primary endpoint of PFS, with a hazard ratio of 0.45. The median PFS for patients treated with niraparib was 9.3 months, compared to 3.9 months for control (p \textless{} 0.0001).

The most common (≥10\%) treatment-emergent grade 3/4 adverse events among all patients treated with niraparib were thrombocytopenia (28.3\%), anemia (24.8\%) and neutropenia (11.2\%). Adverse events were generally managed via dose modifications. The discontinuation rate was 14.7\% for niraparib treated patients and 2.2\% for control. The rates of MDS/AML in the niraparib (1.3\%) and control (1.2\%) arms were similar. There were no deaths among patients during study treatment.

``The majority of women who are diagnosed with advanced ovarian cancer will experience a relapse of their disease, even if they respond to their initial chemotherapy,'' said Dr. Tom Herzog, M.D., Clinical Director, University of Cincinnati Cancer Institute and Professor, Department of Obstetrics and Gynecology at the University of Cincinnati. ``New treatment options are needed to extend the time in between cycles of platinum-based chemotherapy for these patients, and the results from the NOVA study suggest that niraparib could represent an important new treatment option for many patients with ovarian cancer.''

``While the identification of mutations in the BRCA genes was a significant advancement, ovarian cancer remains the deadliest of gynecologic cancers, and new diagnostic and therapeutic options are needed,'' said David Barley, Chief Executive Officer of the National Ovarian Cancer Coalition. ``The results of the NOVA trial are encouraging for patients and their families, and we look forward to seeing the full results of this study this fall.''

About the Phase 3 NOVA Clinical Trial of Niraparib
NOVA is a double-blind, placebo-controlled, international Phase 3 trial of niraparib that planned to enroll 490 patients with recurrent ovarian cancer who were in a response to their most recent platinum-based chemotherapy. Patients were enrolled into one of two independent cohorts based on germline BRCA mutation status. One cohort enrolled patients who were germline BRCA mutation carriers (gBRCAmut), and the second cohort enrolled patients who were not germline BRCA mutation carriers (non-gBRCAmut). The non-gBRCAmut cohort included patients with HRD-positive tumors, including those with somatic BRCA mutations and other HR defects, and patients with HRD-negative tumors. Within each cohort, patients were randomized 2:1 to receive niraparib or placebo and were treated continuously with placebo or 300 milligrams of niraparib, dosed as three 100 milligram tablets once per day, until progression. The primary endpoint of this study was progression-free survival (PFS). Secondary endpoints include patient-reported outcomes, chemotherapy-free interval length, PFS2, overall survival, and other measures of safety and tolerability.  More information about this trial is available at \url{http://clinicaltrials.gov/show/NCT01847274}.

Niraparib is an investigational agent and, as such, has not been approved by the U.S. FDA or any other regulatory agencies.

About Homologous Recombination Deficiency (HRD)
Homologous recombination deficiency (HRD) is a defect in high-fidelity, double-strand DNA repair. HRD results from a variety of causes, including mutations in BRCA and other genes involved in DNA repair, as well as other unidentified causes.  In cells with HRD, such as BRCA1 and BRCA2 mutant cells, lack of functional DNA repair pathways, including PARP, leads to irreparable double-strand breaks, genomic instability, and ultimately cell death. Because HRD-positive cells are sensitive to DNA-damaging processes, they are reliant upon proper DNA repair by other pathways, including PARP-dependent pathways.

Comprehensive assessment of HRD status includes both tBRCA mutational analysis and assessment of genomic instability through the combined analysis of HRD biomarker components LOH, TAI, and LST. myChoiceHRD® is a registered trademark of Myriad Genetics, Inc.

About Niraparib
Niraparib is an oral, once-daily PARP inhibitor that is currently being evaluated in three ongoing pivotal trials. TESARO is building a robust niraparib franchise by assessing activity across multiple tumor types and by evaluating several potential combinations of niraparib with other therapeutics. The ongoing development program for niraparib includes the Phase 3 trial in patients with ovarian cancer (the NOVA trial) as described above; a registrational Phase 2 treatment trial in patients with ovarian cancer (the QUADRA trial); a Phase 3 trial for the treatment of patients with BRCA-positive breast cancer (the BRAVO trial); and a Phase 3 trial in patients with first-line ovarian cancer (the PRIMA trial). Several collaborator-sponsored studies are also underway, including combination trials of niraparib plus pembrolizumab and bevacizumab. Janssen Biotech has licensed rights to develop and commercialize niraparib specifically for patients with prostate cancer worldwide, except in Japan.

About Ovarian Cancer
Approximately 22,000 women are diagnosed each year with ovarian cancer in the United States, and nearly 80\% are diagnosed after the disease has become symptomatic and has progressed to a late stage. Ovarian cancer is the fifth most frequent cause of cancer death among women. Despite high response rates to platinum-based chemotherapy in the second-line advanced treatment setting, 90\% of patients will experience recurrence within two years. If approved, niraparib may address the difficult ``watchful waiting'' periods experienced by patients with recurrent ovarian cancer in between cycles of platinum-based chemotherapy.


\chapter{Anacor Pharmaceuticals, Inc}
\label{chapters/chapter4:anacor}\label{chapters/chapter4::doc}\label{chapters/chapter4:anacor-pharmaceuticals-inc}
On May 16, \emph{Pfizer} announced that it was acquiring \emph{Anacor Pharmaceuticals} (NASDAQ:ANAC) for the hefty sum of \$99.25 in cash per share, or a 55\% premium to where Anacor shares closed on Friday. The crown jewel of the \$5.2 billion acquisition is crisaborole, a non-steroidal topical anti-inflammatory PDE-4 inhibitor that's been submitted for regulatory review in the U.S. for mild-to-moderate atopic dermatitis (a type of eczema). Crisaborole is also being studied as a treatment for psoriasis. Pfizer believes that Anacor's lead compound could generate up to \$2 billion in peak annual sales.

``Crisaborole is a differentiated asset with compelling clinical data that, if approved, has the potential to be an important first-line treatment option for these patients and the physicians who treat them,'' said Albert Bourla, Group President of Pfizer's Global Innovative Pharma and Global Vaccines, Oncology, and Consumer Health Businesses.

That ``compelling clinical data'' Bourla speaks of comes from the AD-301 and AD-302 phase 3 studies released in mid-July 2015 that showed a statistically significant advantage in clearing the chronic rashes that occur with atopic dermatitis compared to the placebo. In terms of primary endpoint, the percentage of patients experiencing an Investigator's Static Global Assessment (ISGA) score of 0 (clear) or 1 (almost clear) with a minimum two-grade drop at day 29 was 32.8\% in AD-301 and 31.4\% in AD-302 compared to the placebo's 25.4\% and 18\% respective effectiveness.

The secondary endpoint, which examined which patients achieved an ISGA of 0 or 1 regardless of whether or not they had a minimum two-grade drop, also demonstrated success for crisaborole. In AD-301 and AD-302, 51.7\% and 48.5\% of patients experience full or almost-full clearing, which compares to 40.6\% and 29.7\% full or almost-full clearing, respectively, for the placebo.

As icing on the cake, Pfizer also gains access to topical toenail fungal treatment Kerydin, which was approved by the Food and Drug Administration in July 2014.

Pfizer believes the deal will not materially affect its outlook in 2016, that it'll be slightly dilutive to full-year EPS in 2017, and be accretive to its bottom-line in 2018 and each year thereafter.

Sounds like a great deal, right? I'm not so sure.
Pfizer may have vastly overpaid for Anacor

While I'm all for having Pfizer use its cash flow to boost the inorganic growth side of the equation, I'd contend that it vastly overpaid for Anacor when it offered \$5.2 billion for the drug developer.
\begin{figure}[htbp]
\centering

\noindent\sphinxincludegraphics{{Crisaborole}.png}
\end{figure}
\begin{figure}[htbp]
\centering

\noindent\sphinxincludegraphics{{Crisaborole1}.png}
\end{figure}

Taking this step by step, Pfizer really is getting two assets: kerydin and cirsaborole. Anacor does have other topical anti-inflammatory products in development, but they're all in the discovery or preclinical stages of development. Aside from these two therapies, the only other drug in clinical studies is AN3365 for infections caused by Gram-negative bacteria, and it's far too early to tell if this clinical therapeutic is effective.

Kerydin is, to be blunt, an afterthought in this acquisition. Anacor forged an agreement with Sandoz to help market Kerydin back in 2014, and last year total distribution and commercialization segment revenue was \$69.7 million. With peak sales estimates of \$400 million, it's not going to move the needle much for Pfizer.

The bigger concern would be for crisaborole. On one hand, there's probably a better than 50-50 shot at FDA approval come its PDUFA date in January 2017 thanks to the drug's meeting its primary and secondary endpoints in both studies and its being generally well tolerated by patients. With few options in treating atopic dermatitis, crisaborole could gobble up market share quickly. But, it's what happens one or two years from now when crisaborole is facing a bounty of potential new competitors that worries me.

Celgene's (NASDAQ:CELG) oral PDE-4 inhibitor Otezla is already approved to treat psoriatic arthritis and plaque psoriasis, but Celgene has hopes of eventually gaining a label expansion for atopic dermatitis as well. In a previously conducted, though small, study involving Otezla that defined treatment benefit as a 50\% (or higher) decrease in the Eczema Area and Severity Index (EASI), Otezla delivered a 62\% success rate. Keep in mind that EASI and ISGA aren't comparable measurements, so we can't simply say one drug is better than the other. But it does suggest that Otezla could be on track to become the first oral eczema treatment for those with moderate-to-severe forms of the disease.

In addition, Regeneron Pharmaceuticals (NASDAQ:REGN) and Sanofi (NYSE:SNY) are expected to file for regulatory approval of injectable dupilumab for the treatment of moderate-to-severe atopic dermatitis in the third quarter. In the LIBERTY AD SOLO1 and SOLO2 trials 37\% and 36\% of patients who an IGA score of 0 or 1 (clear or nearly clear) compared to just 10\% and 8.5\% for the placebo. EASI improvement from baseline was also a healthy 72\% and 69\% for dupilumab compared to just 38\% and 31\%, respectively, for the placebo. Regeneron and Sanofi's injection was also well-tolerated.
Sny Fb

Roche, AstraZeneca, and Chugai in Japan are also working on midstage atopic dermatitis therapies. This is an increasingly crowded space, and \$2 billion seems like a longshot with other successful therapies making their way down the pipeline.

Even with the assumption that crisaborole becomes a blockbuster (\$1 billion in annual sales), it could be a very long time before Pfizer realizes a ``gain'' on its investment. Assuming a healthy margin on crisaborole of say 70\%, and taking into account added revenue from Kerydin, the dilutive effect this acquisition could have on 2017 EPS, and the likelihood that crisaborole would take a few years to ramp up sales, it might be 2024 or 2025 before Pfizer finds its \$5.2 billion ``investment'' in Anacor yielding positive results.

The good news here is Pfizer is generating more than enough cash flow to facilitate additional deals, and its oncology segment is on fire with an immuno-oncology offering (avelumab) waiting on the wings. The bad news is I don't believe its acquisition of Anacor was a particularly good move, and I don't see any immediate benefits to this deal for shareholders.


\chapter{Cerulean Pharma Inc}
\label{chapters/chapter5:ceru}\label{chapters/chapter5::doc}\label{chapters/chapter5:cerulean-pharma-inc}
Cerulean Pharma Inc. (NASDAQ: CERU) announced that the U.S. Food and Drug Administration (FDA) granted Fast Track designation for Cerulean’s lead nanoparticle-drug conjugate, CRLX101, in combination with paclitaxel, for the treatment of platinum-resistant ovarian carcinoma, fallopian tube or primary peritoneal cancer.

“We appreciate the FDA’s acknowledgement of CRLX101’s potential in an area of significant unmet medical need,” said Christopher D. T. Guiffre, President and Chief Executive Officer of Cerulean. “We are encouraged by the profound treatment effect observed early in the ongoing clinical trial with the GOG Foundation, Inc. (GOG), and we look forward to working closely with the FDA as we endeavor to bring a new treatment option to women living with platinum-resistant ovarian cancer.”

CRLX101 is being evaluated in combination with weekly paclitaxel for the treatment of recurrent platinum-resistant ovarian carcinoma in a Phase 1b/2 clinical trial. Data from the Phase 1b portion of the trial were the subject of an oral presentation at the Gynecologic Oncology 2016 Conference in May. These data showed that five of the first nine patients (56\%) enrolled in the trial achieved partial responses. Of note, five of the nine patients enrolled in the Phase 1b trial previously failed Avastin® (bevacizumab) and three of these five patients achieved partial responses. Cerulean is conducting this trial in collaboration with the GOG and expects to provide an update at the European Society for Medical Oncology 2016 Congress.

In 2015, CRLX101 was granted Orphan Drug designation for the treatment of ovarian cancer.

The FDA’s Fast Track Program is designed to facilitate the development and expedite the review of new drugs that are intended to treat serious conditions and that demonstrate the potential to address unmet medical needs. Drugs that receive this designation benefit from more frequent communications and meetings with FDA to review the drug’s development plan, including the design of the proposed clinical trials and the extent of data needed for approval.


\chapter{Medivation}
\label{chapters/chapter6:medivation}\label{chapters/chapter6::doc}\label{chapters/chapter6:id1}
Medivation is an American biopharmaceutical giant which is based in San Francisco.
The company is focused on rapidly developing drugs which help in treating serious diseases with limited treatment possibilities.
The corporation is enlisted on NASDAQ stock exchange under the symbol MDVN.
It has production facilities based in United States along with many Research and Development bases too.

David Hung serves the company as its Chairman and CEO and also leads its R and D department.
As of 2015, it is one of the most commercially successful new companies to emerge in the United States.
The corporation is just a year over a decade old and already has become the most sought after pharmaceutical player in the North American continent. It was founded by David Hung in December 2004 as a result of the acquisition of Medivation Neurology, Inc. Since then, it has collaborated with many other companies from the same industry in order to tighten its grip on the American as well as international markets and also to enhance the workings of its R and D department. Recently it was in the news for collaborating with the European pharmaceutical giant Astellas when the U.S. Food and Drug Administration (FDA) approved the result of its joint venture that produced XTANDI (enzalutamide) capsules. This was a major breakthrough for both the entities as is known to be highly successful in treating patients with metastatic castration-resistant prostate cancer. The procedure of the drug is basically applied in post-chemotherapy of the patients.

Product Trialsmedivation1The development, manufacturing and commercialization of XTANDI will be done in collaboration with Astellas Pharma Inc. in the future, while License and supply agreement has been done in collaboration with CureTech for development, manufacturing and commercialization of its biologic molecules. Medivation’s clinical trials as far as prostate cancer is concerned include STRIVE, which currently stands in Phase II clinical trial. AFFIRM and PREVAIL have already completed the enrollment for Phase III clinical trial. PROSPER and EMBARK are in an ongoing enrollment for the Phase III clinical trial. TERRAIN has done so too for the Phase II clinical trial. Apart from these, AR+, ER+ or PgR+, HER2 Normal and TNBC; and HER2 Amplified and AR+ are part of its breast cancer related Phase II clinical trials.Upheavals \& AchievementsAbout four years back the company was involved in a lawsuit that was filed against it, when its stocks dropped by a staggering 67\% in just one day on its total number of 45 million shares. Izard Nobel LLP was responsible for initiating the case and it carried on for a total four years ending with the decision going in Medivation’s favor and the company getting cleared of all charges. Medivation has managed to win back its reputation since and has managed to establish a firm hold on the market too.

Monday, August 22, 2016 - 6:45am
EDT
Pfizer Inc. (NYSE:PFE) and Medivation, Inc. (NASDAQ:MDVN) today announced that they have entered into a definitive merger agreement under which Pfizer will acquire Medivation, a biopharmaceutical company focused on developing and commercializing small molecules for oncology, for \$81.50 a share in cash for a total enterprise value of approximately \$14 billion. The Boards of Directors of both companies have unanimously approved the merger, which is expected to be immediately accretive to Pfizer’s Adjusted Diluted EPS upon closing, approximately \$0.05 accretive in the first full year after close with additional accretion and growth anticipated thereafter. Pfizer does not expect the transaction to impact its current 2016 financial guidance.

“The proposed acquisition of Medivation is expected to immediately accelerate revenue growth and drive overall earnings growth potential for Pfizer,” said Ian Read, chairman and chief executive officer, Pfizer. “The addition of Medivation will strengthen Pfizer’s Innovative Health business and accelerate its pathway to a leadership position in oncology, one of our key focus areas, which we believe will drive greater growth and scale of that business over the long-term. This transaction is another example of how we are effectively deploying our capital to generate attractive returns and create shareholder value.”

Medivation’s portfolio includes XTANDI® (enzalutamide), an androgen receptor inhibitor that blocks multiple steps in the androgen receptor signaling pathway within the tumor cell. XTANDI is the leading novel hormone therapy in the United States today and generated approximately \$2.2 billion in worldwide net sales over the past four quarters, as recorded by Astellas Pharma Inc., with whom Medivation entered an agreement in 2009 to develop XTANDI globally and commercialize jointly in the U.S. Since its approval for advanced metastatic prostate cancer by the U.S. Food and Drug Administration in 2012, XTANDI has treated 64,000 men to date in the U.S. alone. Medivation and Astellas have built a robust development program for XTANDI, including two Phase 3 studies in non-metastatic prostate cancer and another Phase 3 study in hormone-sensitive prostate cancer. It is also being further developed in Phase 2 studies for the potential treatment of advanced breast cancer and hepatocellular carcinoma.

In addition, Medivation has a promising, wholly-owned, late-stage oncology pipeline, which includes two development-stage oncology assets, talazoparib and pidilizumab. Talazoparib, currently in a Phase 3 study for the treatment of BRCA-mutated breast cancer, has the potential to be a highly potent PARP inhibitor and could be efficacious across several additional tumors. Pidilizumab is an immuno-oncology (IO) asset being developed for diffuse large B-cell lymphoma and other hematologic malignancies and has the potential to be combined with IO therapies in Pfizer’s portfolio.

“We believe the combination with Pfizer is the right next step in our growth trajectory and is a testament to the passion and dedication by which the Medivation team has delivered on our mission to profoundly transform patients’ lives through medically innovative therapies,” said David Hung, M.D., founder, president and CEO of Medivation. “This compelling transaction will deliver significant and immediate value to our stockholders and provides new opportunities for our employees as part of a larger company. We believe that Pfizer is the ideal partner to extend the reach of our blockbuster XTANDI franchise and take our promising, late-stage assets – talazoparib and pidiluzimab – to their next stages of development so that they can be made available to patients as quickly as possible.”

“The proposed acquisition of Medivation will build upon Pfizer’s success with our IBRANCE® (palbociclib) launch in HR+/HER2- metastatic breast cancer and with our strong immuno-oncology portfolio, and will transform Pfizer into a leading oncology company,” said Albert Bourla, group president, Pfizer Innovative Health. “IBRANCE and XTANDI are anchor brands in breast and prostate cancer respectively, giving Pfizer leadership in two hormone-driven cancers. Similar to IBRANCE in the breast cancer setting, XTANDI is being explored for its potential to move from metastatic prostate cancer to treat earlier stages of non-metastatic prostate cancer. In addition, Medivation’s portfolio within prostate cancer and across diverse tumors will complement Pfizer’s broad IO portfolio. Finally, Medivation adds commercial scale to better compete with other top tier oncology companies in advance of the potential emergence of Pfizer’s IO pipeline expected in the next few years. Together, we believe Pfizer and Medivation can bring the full force of our combined research and resources to combat two of the most common cancers, as well as speed cures and make accessible breakthrough medicines to patients, redefining life with cancer.”

Cancer remains the second leading cause of death in the U.S. and a “Top 10” killer worldwide. According to the American Cancer Society, breast cancer and prostate cancer are among the top three cancers by annual incidence in the U.S. There are several parallels between breast and prostate cancer, including the incidence of prostate cancer in the U.S., which is similar to that of breast cancer with approximately 280,000 cases per year.

Pfizer expects to finance the transaction with existing cash.

Under the terms of the merger agreement, a subsidiary of Pfizer will commence a cash tender offer to purchase all of the outstanding shares of Medivation common stock for \$81.50 per share, net to the seller in cash, without interest, subject to any required withholding of taxes. The closing of the tender offer is subject to customary closing conditions, including U.S. antitrust clearance and the tender of a majority of the outstanding shares of Medivation common stock. The merger agreement contemplates that Pfizer will acquire any shares of Medivation that are not tendered into the offer through a second-step merger, which will be completed promptly following the closing of the tender offer. Pfizer expects to complete the acquisition in the Third- or Fourth-Quarter 2016.

Pfizer’s financial advisors for the transaction were Guggenheim Securities and Centerview Partners, with Ropes \& Gray LLP acting as its legal advisor. J.P. Morgan Securities and Evercore served as Medivation’s financial advisors, while Cooley LLP and Wachtell, Lipton, Rosen \& Katz served as its legal advisors.


\chapter{Indices and tables}
\label{index:indices-and-tables}\begin{itemize}
\item {} 
\DUrole{xref,std,std-ref}{genindex}

\item {} 
\DUrole{xref,std,std-ref}{modindex}

\item {} 
\DUrole{xref,std,std-ref}{search}

\end{itemize}



\renewcommand{\indexname}{索引}
\printindex
\end{document}
